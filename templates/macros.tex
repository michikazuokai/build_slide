%----------------------------------------------------------------------------------------
% macros.tex: パス定義・独自コマンド・ロジック制御
%----------------------------------------------------------------------------------------

% --- 1. パスと変数設定 ---
% Python (build_slides2.py) からプレースホルダが置換されます
\newcommand{\assetpath}{@@sourcedir@@}
\newcommand{\pgfpath}{\assetpath/\detokenize{@@sdir@@}/images/} 
\newcommand{\codedir}{\assetpath/\detokenize{@@sdir@@}} 

% 画像検索パスの設定
%  \graphicspath{{images/}{\assetpath/\detokenize{@@sdir@@}/images/}{../project_assets/images/}{../project_assets/emoji/emoji_pngs/}} 
\graphicspath{{../}{../images/}{@@tool_img@@/}{@@emoji_img@@/}{\assetpath/\detokenize{@@sdir@@}/images/}}

% フッター用テキスト
%\newcommand{\myfootertext}{@@sdir@@} 
\newcommand{\myfootertext}{\detokenize{@@sdir@@}}


% --- 2. ラインマーカー(ハイライター)コマンド ---
% 通常のマーカー
\newcommand{\markline}[2][yellow]{% 
  \tikz[baseline=(X.base)]{% 
    \node[inner sep=0pt,outer sep=0pt] (X) {#2}; % 
    \begin{scope}[on background layer] % 
      \fill[#1, opacity=0.35, rounded corners=0.8pt] % 
        ([xshift=-0.15em,yshift=0.00ex]X.south west) rectangle % 
        ([xshift= 0.15em,yshift=2.15ex]X.south east); % 
    \end{scope} % 
  }% 
}

% 線の太さ・位置を調整できる高度なマーカー
\newcommand{\marklineA}[4][yellow]{% 
  \tikz[baseline=(X.base)]{% 
    \node[inner sep=0pt,outer sep=0pt] (X) {#4}; % 
    \begin{scope}[on background layer] % 
      \fill[#1, opacity=0.35, rounded corners=0.8pt] % 
        ([xshift=-0.15em,yshift=#2]X.south west) rectangle % 
        ([xshift= 0.15em,yshift=#3]X.south east); % 
    \end{scope} % 
  }% 
}


% --- 3. 教師用モード制御(奇数ページ開始ロジック) ---
\newif\ifoddslideenforce 
\oddslideenforcefalse   % デフォルトOFF 

\newif\ifoddslideguard 
\oddslideguardfalse 

% 偶数ページなら空白を入れて奇数に戻すコマンド
\newcommand{\ensureoddslide}{% 
  \ifoddslideguard\relax\else 
    \oddslideguardtrue 
    \ifodd\value{page}\relax 
      % 何もしない 
    \else 
      \begin{frame}[plain,noframenumbering] 
        \note{}% 
      \end{frame} 
    \fi 
    \oddslideguardfalse 
  \fi 
}

% frame開始直前に自動挿入
\BeforeBeginEnvironment{frame}{% 
  \ifoddslideenforce 
    \ensureoddslide 
  \fi 
}


% --- 4. ノート関連 (note / noteT) の安全化 ---
\providecommand{\notetitletext}{} 
\providecommand{\noteT}[2]{}      % デフォルトは何もしない 

% frame開始ごとにタイトル変数をクリア
\AtBeginEnvironment{frame}{\gdef\notetitletext{}} 


% --- 5. その他マクロの読み込み ---
\newcommand{\emjMG}{\raisebox{-0.8ex}{\includegraphics[height=3ex]{MG.png}}}  %🔍
\newcommand{\emjblueDa}{\raisebox{-0.8ex}{\includegraphics[height=3ex]{bluediamond.png}}} %🔹
\newcommand{\emjbooka}{\raisebox{-0.8ex}{\includegraphics[height=3ex]{booka.png}}}
\newcommand{\emjbookb}{\raisebox{-0.8ex}{\includegraphics[height=3ex]{bookb.png}}}
\newcommand{\emjbox}{\raisebox{-0.8ex}{\includegraphics[height=3ex]{box.png}}}
\newcommand{\emjbrain}{\raisebox{-0.8ex}{\includegraphics[height=3ex]{brain.png}}} %🧠
\newcommand{\emjblackTa}{\raisebox{-0.8ex}{\includegraphics[height=3ex]{blacktrianglea.png}}} % ▶︎
\newcommand{\emjblackTb}{\raisebox{-0.8ex}{\includegraphics[height=3ex]{blacktriangleb.png}}} % ◀︎
\newcommand{\emjbublea}{\raisebox{-0.8ex}{\includegraphics[height=3ex]{bublex1.png}}}
\newcommand{\emjcalender}{\raisebox{-0.8ex}{\includegraphics[height=3ex]{calenderx.png}}}
\newcommand{\emjcaution}{\raisebox{-0.8ex}{\includegraphics[height=3ex]{caution.png}}}
\newcommand{\emjcheck}{\raisebox{-0.8ex}{\includegraphics[height=3ex]{checkx.png}}}
\newcommand{\emjclipboard}{\raisebox{-0.8ex}{\includegraphics[height=3ex]{clipboard.png}}}
\newcommand{\emjclip}{\raisebox{-0.8ex}{\includegraphics[height=3ex]{clip.png}}}
\newcommand{\emjdoca}{\raisebox{-0.8ex}{\includegraphics[height=3ex]{docx1.png}}}
\newcommand{\emjdocb}{\raisebox{-0.8ex}{\includegraphics[height=3ex]{docx2.png}}}
\newcommand{\emjfacea}{\raisebox{-0.8ex}{\includegraphics[height=3ex]{facea.png}}} %🤔
\newcommand{\emjfoldera}{\raisebox{-0.8ex}{\includegraphics[height=3ex]{foldera.png}}}
\newcommand{\emjgarbage}{\raisebox{-0.8ex}{\includegraphics[height=3ex]{garbage.png}}}
\newcommand{\emjgcap}{\raisebox{-0.8ex}{\includegraphics[height=3ex]{gcap.png}}}
\newcommand{\emjgear}{\raisebox{-0.8ex}{\includegraphics[height=3ex]{gear.png}}}
\newcommand{\emjglobe}{\raisebox{-0.8ex}{\includegraphics[height=3ex]{globe.png}}}
\newcommand{\emjgstar}{\raisebox{-0.8ex}{\includegraphics[height=3ex]{glowingstar.png}}}
\newcommand{\emjlink}{\raisebox{-0.8ex}{\includegraphics[height=3ex]{link.png}}}
\newcommand{\emjloop}{\raisebox{-0.8ex}{\includegraphics[height=3ex]{loopx.png}}}
\newcommand{\emjmap}{\raisebox{-0.8ex}{\includegraphics[height=3ex]{map.png}}}
\newcommand{\emjmato}{\raisebox{-0.8ex}{\includegraphics[height=3ex]{target.png}}}
\newcommand{\emjmemo}{\raisebox{-0.8ex}{\includegraphics[height=3ex]{memo.png}}}
\newcommand{\emjng}{\raisebox{-0.8ex}{\includegraphics[height=3ex]{ng.png}}}
\newcommand{\emjok}{\raisebox{-0.8ex}{\includegraphics[height=3ex]{ok.png}}}
\newcommand{\emjpallet}{\raisebox{-0.8ex}{\includegraphics[height=3ex]{palettex1.png}}}
\newcommand{\emjpca}{\raisebox{-0.8ex}{\includegraphics[height=3ex]{pc.png}}} %💻
\newcommand{\emjpc}{\raisebox{-0.8ex}{\includegraphics[height=3ex]{monitor.png}}}  %🖥️
\newcommand{\emjpen}{\raisebox{-0.8ex}{\includegraphics[height=3ex]{pen.png}}}
\newcommand{\emjpiece}{\raisebox{-0.8ex}{\includegraphics[height=3ex]{piece.png}}}
\newcommand{\emjpiny}{\raisebox{-0.8ex}{\includegraphics[height=3ex]{piny.png}}}
\newcommand{\emjpin}{\raisebox{-0.8ex}{\includegraphics[height=3ex]{pinx.png}}}
\newcommand{\emjrocket}{\raisebox{-0.8ex}{\includegraphics[height=3ex]{rocket.png}}}
\newcommand{\emjryubi}{\includegraphics[height=3ex]{r_yubi.png}}  %👉
\newcommand{\emjsankaku}{\raisebox{-0.8ex}{\includegraphics[height=3ex]{三角.png}}}
\newcommand{\emjstopwatch}{\raisebox{-0.8ex}{\includegraphics[height=3ex]{stopwatch.png}}}
\newcommand{\emjtesttube}{\raisebox{-0.8ex}{\includegraphics[height=3ex]{testtubex1.png}}}
\newcommand{\emjtoola}{\raisebox{-0.8ex}{\includegraphics[height=3ex]{toolx2.png}}} %🛠️
\newcommand{\emjuyubi}{\raisebox{-0.8ex}{\includegraphics[height=3ex]{uyubi.png}}}  %👆
\newcommand{\emjgroupa}{\raisebox{-0.8ex}{\includegraphics[height=3ex]{groupa.png}}} % 👥
\newcommand{\emjredtriangle}{\raisebox{-0.8ex}{\includegraphics[height=3ex]{点2.png}}} %
\newcommand{\emjtelescorpe}{\raisebox{-0.8ex}{\includegraphics[height=3ex]{telescorpe.png}}} % 🔭
\newcommand{\emjwrench}{\raisebox{-0.8ex}{\includegraphics[height=3ex]{wrench.png}}} % 🔧
\newcommand{\emjkey}{\raisebox{-0.8ex}{\includegraphics[height=3ex]{key.png}}} % 🔑
\newcommand{\emjbang}{\raisebox{-0.8ex}{\includegraphics[height=3ex]{bang.png}}} % ❗️
\newcommand{\emjspeechB}{\raisebox{-0.8ex}{\includegraphics[height=3ex]{speechB.png}}} % ❗️
\newcommand{\emjexplosion}{\raisebox{-0.8ex}{\includegraphics[height=3ex]{explosion.png}}} % 💥
\newcommand{\emjbanzai}{\raisebox{-0.8ex}{\includegraphics[height=3ex]{banzai.png}}} % 🙌
\newcommand{\emjdotblack}{\raisebox{-0.8ex}{\includegraphics[height=3ex]{dotblack.png}}} % ⚫️
%%%\newcommand{\emjdotblackS}{\raisebox{0ex}{\includegraphics[height=1.5ex]{dotblack.png}}} % ⚫️
\newcommand{\emjdotblackS}{\raisebox{0ex}{\includegraphics[height=1.5ex]{dotblack.png}}\hspace{-0em}}