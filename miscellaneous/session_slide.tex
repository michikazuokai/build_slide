% @@@--(metropolis)--@@@
%----------------------------------------------------------------------------------------
%    TITLE PAGE
%----------------------------------------------------------------------------------------

\title{08 セッションと状態の保存}
\subtitle{Flaskのsessionの使い方}

\date{}

%----------------------------------------------------------------------------------------
%    PRESENTATION SLIDES
%----------------------------------------------------------------------------------------

\begin{document}

\begin{frame}
    % Print the title page as the first slide
    \titlepage
\end{frame}

%------------------------------------------------
\section{First Section}
%------------------------------------------------

%@@@@@@@@@@@@@@@@@@@@@@

%----------------------------------------------------------------------------------------
%    page
%----------------------------------------------------------------------------------------
\begin{frame}[fragile]{セッションとは?なぜ必要?}
\includegraphics[height=2ex]{point1.png} 状態を保持するとは? 
\vspace{1em}

Webアプリ(HTTP)は \\
\textbf{「ステートレス(stateless)」=前のやり取りを覚えない}のが基本です。

\bigskip
\includegraphics[height=2ex]{point1.png} 例:

Aさんが /login にログイン  \\
\includegraphics[height=2ex]{arrow1.png} /home に移動 \\
\includegraphics[height=2ex]{arrow1.png} 本来は「誰だったか?」を覚えていない \\
\includegraphics[height=2ex]{arrow1.png} でも、ログインしたままで使いたい! というニーズがある \\
\end{frame}

%----------------------------------------------------------------------------------------
%    page
%----------------------------------------------------------------------------------------
\begin{frame}[fragile]{状態が保持できないとどうなるか?}
\begin{lstlisting}[language=Python, basicstyle=\ttfamily\small, frame=single]
1. 名前をフォームで送る(例:山田)
2. サーバが受け取って「こんにちは、山田さん!」と表示
3. でもページを切り替えると「誰だっけ?」と忘れてしまう
\end{lstlisting}


\includegraphics[height=2ex]{arrow1.png} 解決したいこと:
\bigskip
\begin{itemize}
  \setlength{\itemsep}{1em}  % 行間(項目間)の間隔を広げる
  \item ログイン中の「名前」
  \item カートに入れた「商品」
  \item 訪問回数やアクセス情報
\end{itemize}
\bigskip
これらを\textbf{一時的に記憶しておく仕組み}が必要!

\end{frame}

%----------------------------------------------------------------------------------------
%    page
%----------------------------------------------------------------------------------------
\begin{frame}[fragile]{どうやって記憶する? → 2つの方法がある}

① ユーザーのブラウザ側に保存する方法 \includegraphics[height=2ex]{arrow1.png} クッキー \\
\vspace{1em}

② サーバ(アプリ)側に保存する方法 \includegraphics[height=2ex]{arrow1.png}  セッション  \\
\bigskip
この2つが「状態を保持する代表的な仕組み」です。
\end{frame}

%----------------------------------------------------------------------------------------
%    page
%----------------------------------------------------------------------------------------
\begin{frame}[fragile]{セッションとクッキーの違い}

\includegraphics[width=0.7\textwidth]{cookie2.png}
\bigskip

補足:Flaskのセッションは「セッションIDだけをクッキーに保存」
\bigskip

\includegraphics[height=2ex]{arrow1.png} 実際の中身は Flask 側で管理され、安全です(暗号化も可)
\end{frame}

%----------------------------------------------------------------------------------------
%    page
%----------------------------------------------------------------------------------------
\begin{frame}[fragile]{まとめ:なぜセッションが必要か?}

「誰がログインしているか」「フォームで送った情報を次のページでも使いたい」\\
\includegraphics[height=2ex]{arrow1.png} その\textbf{“覚えておく仕組み”}として、セッションがある
\bigskip

クッキーと違い、Flaskで扱いやすく、安全性が高い
\bigskip

このように、「状態保持」\includegraphics[height=2ex]{arrow1.png} 「記憶が必要な場面」\\
\includegraphics[height=2ex]{arrow1.png} 「解決策としてのセッションとクッキー」
\bigskip

というステップ構造にすると、理解しやすくなります。

\end{frame}

%----------------------------------------------------------------------------------------
%    page
%----------------------------------------------------------------------------------------
\begin{frame}[fragile]{Flaskでのセッションの基本}
\includegraphics[height=2ex]{point1.png} セッションを使う準備
\begin{lstlisting}[language=Python, basicstyle=\ttfamily\small, frame=single]
[python]
from flask import Flask, session

app = Flask(__name__)
app.secret_key = 'secret_key'  # セッション利用に必須
\end{lstlisting}

\includegraphics[height=2ex]{point1.png}  値の保存・取得・削除の基本

\begin{lstlisting}[language=Python, basicstyle=\ttfamily\small, frame=single]
[python]
# 値の保存
session['username'] = '山田'
# 値の取得
username = session.get('username', 'ゲスト')
# 値の削除
session.pop('username', None)
# 全削除(ログアウト時など)
session.clear()
\end{lstlisting}

\end{frame}

%----------------------------------------------------------------------------------------
%    page
%----------------------------------------------------------------------------------------
\begin{frame}[fragile]{「ようこそアプリ」実装}
\includegraphics[height=2ex]{点1.png}  form.html(名前入力フォーム)
\begin{lstlisting}[language=Python, basicstyle=\ttfamily\small, frame=single]
[html]
<form method="POST">
  名前:<input type="text" name="username">
  <button>送信</button>
</form>
\end{lstlisting}

\includegraphics[height=2ex]{点1.png} Flaskコード(app.py)
\begin{lstlisting}[language=Python, basicstyle=\ttfamily\small, frame=single]
[python]
from flask import Flask, render_template, request, session, redirect

app = Flask(__name__)
app.secret_key = 'secret_key'
<<<  続く >>>
\end{lstlisting}
\end{frame}

%----------------------------------------------------------------------------------------
%    page
%----------------------------------------------------------------------------------------
\begin{frame}[fragile]{「ようこそアプリ」実装}

\begin{lstlisting}[language=Python, basicstyle=\ttfamily\small, frame=single]
@app.route("/", methods=["GET", "POST"])
def index():
    if request.method == "POST":
        session['username'] = request.form.get("username")
        return redirect("/welcome")
    return render_template("form.html")

@app.route("/welcome")
def welcome():
    username = session.get("username", "ゲスト")
    return f"ようこそ、{username} さん!"
\end{lstlisting}
\end{frame}

%----------------------------------------------------------------------------------------
%    page
%----------------------------------------------------------------------------------------
\begin{frame}[fragile]{ログアウト処理・セッション削除}
\includegraphics[height=2ex]{点1.png} ログアウト機能追加

\begin{lstlisting}[language=Python, basicstyle=\ttfamily\small, frame=single]
[python]
@app.route("/logout")
def logout():
    session.clear()
    return redirect("/")
\end{lstlisting}

作業手順:

\bigskip
\begin{itemize}
  \setlength{\itemsep}{1em}  % 行間(項目間)の間隔を広げる
  \item 名前を入力 → セッションに保存
  \item /welcome で「ようこそ〜」と表示される
  \item /logout にアクセス → セッションがクリアされて / に戻る
  \item 再度 /welcome にアクセス → 「ようこそ、ゲストさん!」
\end{itemize}
\end{frame}


%----------------------------------------------------------------------------------------
%    page
%----------------------------------------------------------------------------------------
\begin{frame}[fragile]{まとめ}
\includegraphics[width=1.0\textwidth]{まとめ.png} 

\end{frame}
