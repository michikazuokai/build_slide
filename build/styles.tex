%----------------------------------------------------------------------------------------
% styles.tex: テーマ・カラー・レイアウト・装飾の設定
%----------------------------------------------------------------------------------------

% --- 1. Metropolis テーマの基本設定 ---
\usetheme{metropolis}
\metroset{block=fill, sectionpage=progressbar, progressbar=foot}

% 背景色(tech のとき白など): Python (build_slides2.py) で差し込み 
\setbeamercolor{background canvas}{bg=white}


% --- 2. カラーパレットの定義  ---
\definecolor{CanvaGreen}{HTML}{2E7D32}   % メインの濃い緑
\definecolor{PaleGreen}{HTML}{F1F8E9}    % 背景の薄い緑
\definecolor{MyDarkGreen}{HTML}{587a7f}  % 枠線用
\definecolor{DeepText}{HTML}{1C1C1C}     % 本文文字色
\definecolor{MyWhiteBlue}{HTML}{F2FAFB}  % ボックス背景用
\definecolor{BananaColor}{HTML}{FFFD78}  % アクセント
\definecolor{myblue}{HTML}{7488FF}       % 標準ブロック用
\definecolor{mylightblue}{HTML}{E3EEFF}  % 標準ブロック背景用


% --- 3. Beamer 各種テンプレートの再定義  ---

% フッター設定(\scriptsize\color{gray!50} \myfootertext は Python で置換)
\setbeamertemplate{footline}{%
  \leavevmode
  \hbox to \paperwidth{%
    \hspace*{0.2cm}%
    \scriptsize\color{gray!50} \myfootertext%
    \hfill%
    \scriptsize\color{gray} \insertframenumber{} / \inserttotalframenumber%
    \hspace*{0.4cm}%
  }%
  \vspace{1pt}%
}

% フレームタイトル設定 (番号. タイトル)
\setbeamertemplate{frametitle}{%
  \vspace{0.6ex}%
  \begin{beamercolorbox}[wd=\paperwidth,sep=0.5ex,leftskip=0.9em,rightskip=0.5em]{frametitle}%
    \usebeamerfont{frametitle}%
    \insertframenumber.\,\insertframetitle%
  \end{beamercolorbox}%
}

% セクションページのデザイン
\setbeamerfont{section title}{size=\LARGE,series=\bfseries}
\setbeamertemplate{section page}{
  \begin{centering}
    \vfill
    \rule{\linewidth}{2pt}\par
    \vspace{1ex}
    {\usebeamerfont{section title}\Huge\bfseries \insertsection}\par
    \vspace{1ex}
    \rule{\linewidth}{2pt}\par
    \vfill
  \end{centering}
}

% --- セクションページの自動挿入設定(動作の定義) ---
% \section{...} が使われるたびに、自動的に中扉スライドを挿入します
\AtBeginSection[]{
  \begin{frame}[plain,noframenumbering]
    \sectionpage
  \end{frame}
}

% --- 4. tcolorbox のデフォルト設定とカスタムボックス  ---

\tcbset{
    enhanced,
    colback=MyWhiteBlue,
    colframe=MyDarkGreen,
    coltitle=white,
    fonttitle=\bfseries\sffamily\small,
    boxrule=0.5pt,
    arc=1mm,
    sharp corners=south,
    left=3mm, right=3mm,
    top=0.5mm, bottom=0.5mm,
    toptitle=0mm,
    bottomtitle=0mm,
    before skip=0.8em, after skip=0.2em,
    shadow={0mm}{0mm}{0mm}{black!0}
}

% 箇条書き用カスタムボックス
\newtcolorbox{myListbox}[1]{
  enhanced,
  detach title,
  before upper={{\bfseries\large #1}\par\medskip},
  colbacktitle=white, 
  colframe=gray!50,
  colback=white,
  boxrule=1pt,
  fonttitle=\bfseries\color{black},
  title=#1,
  after skip=1.5ex,
  % 内部の箇条書き余白調整
  before upper={
    \setbeamertemplate{itemize ispan}{0pt}
    \setbeamertemplate{itemize items}[default]
    \setlength{\leftmargini}{1.5em}
    \addtobeamertemplate{itemize/enumerate body begin}{}{\setlength{\itemsep}{0pt}\setlength{\parskip}{0pt}}
  }
}


% --- 5. 標準ブロック・exampleblock の調整  ---

\setbeamertemplate{blocks}[rounded]
\setbeamercolor{block title}{bg=myblue, fg=white}
\setbeamercolor{block body}{bg=mylightblue, fg=black}

% exampleblock 専用設定(タイトル白、本文黒、itemize色固定)
\setbeamercolor{block title example}{fg=white}
\setbeamercolor{block body example}{fg=black}

\AtBeginEnvironment{exampleblock}{%
  \setbeamercolor{itemize item}{fg=black}
  \setbeamercolor{itemize subitem}{fg=black}
  \setbeamercolor{item}{fg=black}
}


% --- 6. その他スタイル調整(表の列型など)  ---
\newcolumntype{C}[1]{>{\centering\arraybackslash}p{#1}}
\newcolumntype{M}[1]{>{\raggedright\arraybackslash}m{#1}}


% --- 7. モード切替フラグ(Pythonからの差し込み)  ---
\mypausemodetrue
\teachermodetrue
\setbeameroption{show notes}

\makeatletter
\renewcommand{\noteT}[2]{%
 \gdef\notetitletext{#1}%
 \note{#2}%
}
\renewcommand{\notetitletext}{}%
\setbeamertemplate{note page}{%
 \begin{minipage}{\linewidth}
 \vspace{1.2ex}
 {\Large\bfseries
 \ifx\notetitletext\@empty
 \insertframetitle
 \else
 \notetitletext
 \fi
 }\par
 \vspace{-1.2ex}
 \rule{\linewidth}{0.8pt}\par
 \vspace{0.8ex}
 {\scriptsize \insertnote}
 \end{minipage}
}
\makeatother
\oddslideenforcetrue
